% Autor: Dominik Harmim <harmim6@gmail.com>

\documentclass[a4paper, 10pt, twocolumn]{article}

\usepackage[czech]{babel}
\usepackage[utf8]{inputenc}
\usepackage[T1]{fontenc}
\usepackage[left=2cm, top=2cm, text={17cm, 25cm}]{geometry}
\usepackage[unicode, colorlinks, hypertexnames=false, citecolor=red]{hyperref}
\usepackage{times}
\usepackage{graphicx}

\renewcommand{\tt}[1]{\texttt{\textbf{#1}}}


\begin{document}
	\twocolumn[
		\begin{@twocolumnfalse}
			\begin{center}
				{\Large
					Vysoké učení technické v~Brně \\
					Fakulta informačních technologií \\
				}
				{\includegraphics[width=.4 \linewidth]{img/FIT_logo.pdf}} \\

				{\LARGE
					Bezpečnost informačních systémů \\
					Projekt\,--\,The FITfather \\[.4cm]
				}

				{\large
					Dominik Harmim (xharmi00) \\
					\texttt{xharmi00@stud.fit.vutbr.cz} \\
					\today
				}
			\end{center}
		\end{@twocolumnfalse}
	]


	\section{Úvod}

	Cílem tohoto projektu je získat co nejvíce tajemství ukrytých
	na privátních serverech ve vnitřní síti \tt{bis.fit.vutbr.cz}.
	Následující kapitoly popisují zmapování vnitřní sítě a~postup
	získání jednotlivých tajemství.


	\section{Zmapování sítě}

	Po přihlášení na server \tt{bis.fit.vutbr.cz} bylo příkazem
	\tt{ifconfig} zjištěno, že se počítač nachází v~síti
	192.168.122.0/24. Příkazem \tt{nmap -p- 192.168.122.0/24}
	byly nalezeny následující počíteče a~jejich otevřené porty:
	\begin{itemize}
		\item
			192.168.122.38
			\begin{itemize}
				\item 21/tcp ftp
				\item 22/tcp ssh
				\item 80/tcp http
			\end{itemize}

		\item
			192.168.122.42
			\begin{itemize}
				\item 22/tcp ssh
				\item 111/tcp rpcbind
			\end{itemize}

		\item
			192.168.122.77
			\begin{itemize}
				\item 22/tcp ssh
				\item 111/tcp rpcbind
			\end{itemize}

		\item
			192.168.122.83
			\begin{itemize}
				\item 22/tcp ssh
				\item 111/tcp rpcbind
			\end{itemize}

		\item
			192.168.122.105
			\begin{itemize}
				\item 22/tcp ssh
				\item 80/tcp http
				\item 111/tcp rpcbind
				\item 3306/tcp mysql
			\end{itemize}

		\item
			192.168.122.150
			\begin{itemize}
				\item 22/tcp ssh
				\item 111/tcp rpcbind
			\end{itemize}

		\item
			192.168.122.155
			\begin{itemize}
				\item 22/tcp ssh
				\item 111/tcp rpcbind
			\end{itemize}

		\item
			192.168.122.169
			\begin{itemize}
				\item 22/tcp ssh
				\item 80/tcp http
				\item 111/tcp rpcbind
				\item 42424/tcp unknown
			\end{itemize}

		\item
			192.168.122.206
			\begin{itemize}
				\item 22/tcp ssh
				\item 111/tcp rpcbind
			\end{itemize}

		\item
			192.168.122.215
			\begin{itemize}
				\item 22/tcp ssh
				\item 111/tcp rpcbind
			\end{itemize}

		\item
			192.168.122.220
			\begin{itemize}
				\item 22/tcp ssh
				\item 23/tcp telnet
				\item 80/tcp http
			\end{itemize}

		\item
			192.168.122.227
			\begin{itemize}
				\item 22/tcp ssh
				\item 111/tcp rpcbind
			\end{itemize}
	\end{itemize}


	\section{Získání tajemství}

	Tato kapitola popisuje postup získání jednotlivých tajemství.

	\subsection{Tajemství~A}

	Na počítači 192.168.122.38 běží HTTP server na portu 80. Při
	přístupu přes prohlížeč \tt{elinks} příkazem \tt{elinks
	http://192.168.122.38/} bylo zjištěno, že na serveru běží
	webová aplikace pro zadávání a~vyhledávání zaměstnanců určité
	firmy, která pravděpodobně používá nějakou SQL databázi. Je
	možné zkusit  \textit{SQL injection} vložením řetězce
	\tt{"foo foo;} do vyhledávacího pole. Zobrazí se SQL chybové
	hlášení, ze kterého lze zjistit, že se jedná o~SQL server
	MariaDB a~provádí se dotaz \tt{SELECT id, name, email, address
	FROM contact WHERE name LIKE "\%"foo foo;\%}. Je možné tohoto
	využít pomocí příkazu \tt{UNION} a~do uvedených databázových
	sloupců, které se zobrazují v~aplikaci, je možné vytáhnout
	požadovaná data. Zadáním \tt{" UNION SELECT 42, `table\_name`,
	'foo', 'foo' FROM `information\_schema`.`tables` WHERE
	`table\_type` = 'BASE TABLE';\#} do vyhledávacího pole je
	získán seznam všech definovaných databázových tabulek.
	V~tabulce s~názvem \tt{auth} by mohly být uložené nějaké
	přihlašovací údaje. Názvy sloupců této tabulky lze získat
	zadáním \tt{" UNION SELECT 42, `column\_name`, 'foo',
	'foo' FROM `information\_schema`.`columns` WHERE `table\_name`
	= 'auth';\#} do vyhledávacího pole. Bylo zjištěno, že tato
	tabulka obsahuje sloupce \tt{id}, \tt{login} a~\tt{passwd}.
	Vypsání sloupců \tt{login} a~\tt{passwd} je možné zadáním
	\tt{" UNION SELECT 42, `login`, `passwd`, 'foo' FROM
	`auth`;\#} do vyhledávacího pole. V~poli \tt{passwd} u~sloupce,
	kde má pole \tt{login} hodnotu \tt{admin}, se nachází
	tajemství~\textbf{A}.

	\subsection{Tajemství~B}

	Na serveru 192.168.122.169 běží na portu 42424 neznámá služba.
	Pokusem o~připojení na tento port přes protokol FTP bylo zjištěno,
	že na tomto portu běží FTP server. Po připojení příkazem
	\tt{ftp 192.168.122.169 42424} byly vyžadovány přihlašovací
	údaje. Bylo vyzkoušeno, že je možné se autentizovat jako
	anonymní uživatel s~přihlašovacím jménem \tt{anonymous}
	a~s~prázdným heslem. Na FTP serveru se nachází soubor
	\tt{/secret.txt}. Po jeho stažení příkazem \tt{get /secret.txt}
	a~vypsání příkazem \tt{cat \textasciitilde/secret.txt} bylo
	získáno tajemství~\textbf{B}.

	\subsection{Tajemství~C}

	Na počítači 192.168.122.169 běží HTTP server na portu 80.
	Při přístupu přes prohlížeč \tt{elinks} příkazem \tt{elinks
	http://192.168.122.169/} bylo zjištěno, že lze prohledávat
	dostupné adresáře. Pří přístupu k~souboru
	\tt{http://192.168.122.169/etc/raddb/\\sql.conf} bylo získáno
	tajemství~\textbf{C}.

	\subsection{Tajemství~D}
	\label{sec:secG}

	Na počítači 192.168.122.220 běží SSH server na portu 22. Při
	pokusu o~připojení na tento server přes protokol SSH příkazem
	\tt{ssh 192.168.122.220} byl vypsán řetězec \tt{Hello, smith!}
	a~byly vyžadovány přihlašovací údaje. Při přihlášení jako uživatel
	s~přihlašovacím jménem \tt{smith} příkazem \tt{ssh
	smith@192.168.122.220} už nebylo heslo vyžadováno. Příkazem
	\tt{tcpdump -i ens3 -w out.pcap} byl vygenerován záznam síťové
	komunikace na rozhraní s~názvem \tt{ens3}, který byl zjištěn
	příkazem \tt{ifconfig}. Síťová komunikace byla zaznamenávána
	pouze po určitou dobu, poté byl zaznamenaný výstup uložen do
	souboru. Tento soubor byl následně analyzován v~programu Wireshark.
	Protože na tomto serveru běží služba TELNET na portu 23, tak byla
	analýza zaměřená na pakety právě protokolu TELNET. Při zobrazení
	TCP toku těchto paketů bylo vidět, že se zasílá přihlašovací
	jméno \tt{ada} a~heslo \tt{nachystejteuzenace} v~čisté podobě.
	Při přihlášení na server 192.168.122.220 přes protokol TELNET
	příkazem \tt{telnet 192.168.122.220} s~přihlašovacím jménem
	\tt{ada} a~heslem \tt{nachystejteuzenace} byl na tomto serveru
	nalezen soubor \tt{\textasciitilde/secret.txt}. Po je ho vypsání
	příkazem \tt{cat \textasciitilde/secret.txt} bylo nalezeno
	tajemství~\textbf{D}.

	\subsection{Tajemství~E}

	Na počítači 192.168.122.220 běží HTTP server na portu 80. Při
	přístupu přes prohlížeč \tt{elinks} příkazem \tt{elinks
	http://192.168.122.220/} byl zobrazen přihlašovací formulář, kde
	bylo vyžadováno jméno a~heslo. Příkazem \tt{curl -v
	http://192.168.122.220/} bylo zjištěno, že pro kontrolu přihlášení
	se používá cookie s~názvem \tt{LOGGED\_IN}, které je nastaveno na
	hodnotu \tt{False}. Byl proveden HTTP dotaz na tento server
	s~nastavením tohoto cookie na hodnotu \tt{True} příkazem \tt{curl
	-\,-cookie 'LOGGED\_IN=True' http://192.168.122.220/}. Tímto bylo
	získáno tajemství~\textbf{E}.

	\subsection{Tajemství~F}

	Na počítači 192.168.122.227 běží SSH sever na portu 22. Při pokusu
	o~připojení na tento server přes protokol SSH příkazem \tt{ssh
	192.168.122.227} byla zobrazena hláška, která říká, že je možné
	přihlásit se jako uživatel \tt{teacher} a~byly vyžadovány
	přístupové údaje. Bylo zjištěno, že je možné se přihlásit jako
	uživatel \tt{teacher} s~heslem \tt{teacher} příkazem \tt{ssh
	teacher@192.168.122.227}. Bylo zjištěno, že na tomto serveru
	je možné využít zranitelnosti příkazu \tt{sudo}, kde při zadání
	příkazu s~prefixem \tt{sudo -u\#-1} není vyžadováno heslo správce,
	ale heslo aktuálně přihlášeného uživatele. Proto bylo vyzkoušeno
	na celém serveru vyhledat soubory, které ve svém názvu obsahují
	podřetězec \tt{secret} příkazem \tt{sudo -u\#-1 find / -name
	*secret*}. Byl nalezen soubor \tt{/root/secret.txt}. Vypsáním
	tohoto souboru příkazem \tt{sudo -u\#-1 cat /root/secret.txt}
	bylo získáno tajemství~\textbf{F}.

	\subsection{Tajemství~G}

	Na serveru 192.168.122.38 běží FTP server na portu 21. Při pokusu
	o~připojení na tento server přes protokol FTP příkazem \tt{ftp
	192.168.122.38} byly vyžadovány přístupové údaje. Tímto bylo
	také zjištěno, že se jedná o~FTP server verze \tt{vsFTPd 2.3.4},
	který obsahuje takovou zranitelnost, že při přihlášení
	s~uživatelským jménem, které obsahuje podřetězec \tt{:)}, lze
	zadat prázdné heslo. Při pokusu o~přihlášení stejným příkazem, se
	zadáním uživatelského jména \tt{foo:)} a~prázdného hesla, se
	pouze zobrazí, že je otevřen port 53244. Při přihlášení na
	tento server přes protokol FTP na port 53244 příkazem \tt{ftp
	192.168.122.38 53244} bylo získáno tajemství~\textbf{G}.

	\subsection{Tajemství~H}

	Na serveru 192.168.122.220 (připojení viz tajemství~D,
	kapitola~\ref{sec:secG}) byly vyhledávány v~dostupných adresářích
	soubory, které ve svém názvu obsahují podřetězec \tt{secret}. Příkazem
	\tt{ls /usr/bin/ | grep secret} byl nalezen spustitelný soubor
	\tt{/usr/bin/show-secret}. Spuštěním tohoto souboru příkazem
	\tt{/usr/bin/show-secret} bylo získáno tajemství~\textbf{H}.

	\subsection{Tajemství~I}

	Na počítači 192.168.122.105 běží HTTP server na portu 80. Při
	přístupu přes prohlížeč \tt{elinks} příkazem \tt{elinks
	http://192.168.122.105/} bylo zjištěno, že na serveru existuje
	adresář \tt{www}. Při následném přístupu do tohoto adresáře
	příkazem \tt{elinks http://192.168.122.105/www/} se zobrazuje
	chybová hláška s~kódem 500. Jedná se o~klasické chybové hlášení
	\textit{PHP frameworku Nette}. Zdrojové kódy aplikací napsaných
	v~tomto frameworku jsou typicky uloženy o~úroveň výše v~adresáři
	\tt{app}. Proto byl vyzkoušen přístup do tohoto adresáře příkazem
	\tt{elinks http://192.168.122.105/app/}. Zde je možné procházet
	adresářovou strukturu. Prohledáváním jednotlivých souborů bylo
	v~souboru \tt{http://192.168.122.105/app/config/\\local.neon} jako
	heslo k~databázi nalezeno tajemství~\textbf{I}.

	\subsection{Tajemství~J}

	Na počítači 192.168.122.77 běží SSH server na portu 22. Při pokusu
	o~připojení na tento server přes protokol SSH příkazem \tt{ssh
	192.168.122.77} byly vyžadovány přístupové údaje. Bylo vyzkoušeno,
	že je možné přihlásit se jako uživatel \tt{root} s~heslem \tt{root}
	příkazem \tt{ssh root@192.168.122.77}. Na tomto serveru byl
	nalezen soubor \tt{\textasciitilde/secret.txt}. Jeho vypsáním 
	příkazem \tt{cat \textasciitilde/secret.txt} bylo získáno
	tajemství~\textbf{J}.


	\section{Závěr}

	Všech 10 tajemství (A-F) bylo úspěšně nalezeno, jak je popsáno
	v~kapitolách výše.
\end{document}
